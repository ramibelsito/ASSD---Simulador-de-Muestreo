\section*{Modelado del ZOH (Sample \& Hold)}

\subsection*{Respuesta impulsiva del ZOH}
La respuesta al aplicar un impulso $\delta(t)$ a un bloque ZOH es un pulso rectangular de ancho $T$:
\[
h_{ZOH}(t) =
\begin{cases}
1, & 0 \leq t < T \\
0, & \text{caso contrario}
\end{cases}
\]

\subsection*{Salida ante un tren de impulsos}
Si la entrada es una señal muestreada mediante un tren de impulsos $x^*(t)$, la salida es
\[
y(t) = x^*(t) \otimes h_{ZOH}(t)
\]
donde $\otimes$ denota convolución.

\subsection*{Espectro}
En frecuencia:
\[
Y(\omega) = X^*(\omega) \cdot H_{ZOH}(\omega)
\]

\subsection*{Espectro de la señal muestreada}
La transformada de Fourier del tren de impulsos es
\[
X^*(\omega) = \frac{1}{T} \sum_{m=-\infty}^{\infty} X(\omega - m\omega_s),
\]
con $\omega_s = \tfrac{2\pi}{T}$.

\subsection*{Respuesta en frecuencia del ZOH}
La transformada de Fourier de $h_{ZOH}(t)$ es
\[
H_{ZOH}(\omega) = \int_{0}^{T} e^{-j \omega t} dt 
= \frac{1 - e^{-j\omega T}}{j\omega}
= T \, e^{-j \omega T/2} \, \mathrm{sinc}\!\left(\frac{\omega T}{2}\right).
\]

\subsection*{Espectro de salida}
Por lo tanto,
\[
Y(\omega) =
\left[
\frac{1}{T} \sum_{m=-\infty}^{\infty} X(\omega - m\omega_s)
\right]
\cdot
\Bigg\{ T \, \mathrm{sinc}\!\left(\frac{\omega T}{2}\right) \Bigg\}.
\]

\subsection*{Propiedades}
\begin{itemize}
    \item Los ceros de $H_{ZOH}(\omega)$ ocurren cuando
    \[
    \mathrm{sinc}\!\left(\frac{\omega T}{2}\right) = 0
    \quad \Longrightarrow \quad
    \omega = k \frac{2\pi}{T}, \quad k = \pm 1, \pm 2, \dots
    \]
    \item El espectro resultante consiste en réplicas de $X(\omega)$, moduladas por la envolvente $\mathrm{sinc}\!\left(\tfrac{\omega T}{2}\right)$ y un desfase lineal $e^{-j\omega T/2}$.
\end{itemize}
