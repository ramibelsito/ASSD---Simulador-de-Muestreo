\subsection{Muestreo de Señales}
El objetivo del muestreo de las señales es conseguir obtener una reconstrucción fiel a la señal continua original partiendo de muestras discretas de la misma.
Se debe tener en cuenta que la frecuencia de muestreo ($f_s$, inversa al período que separa las muestras discretas mencionadas previamente) debe ser, como mínimo, el doble de la frecuencias máxima presente en la señal original ($f_{max}$). Esto se conoce como el Teorema de Nyquist-Shannon y se expresa matemáticamente como:

\begin{equation}
    f_s \geq 2 f_{\max}.
    \label{Nyquist}
\end{equation}
Tras este proceso, el espectro de la señal sampleada se repite cada $\pm (N\cdot f_s)$. Es por eso que, luego del muestreo, la señal original no se recupera sino hasta después de aplicar filtros de alta frecuencia que eliminen dichas replicas.
De no cumplirse con el criterio de Nyquist, se produce un fenómeno conocido como \textit{aliasing}, que genera distorsiones en la señal reconstruida debido a que las replicas se ubican dentro del espectro de frecuencia de la señal original.

\subsection{Tipos de Muestreo}
Existen diferentes modelos para describir el proceso de muestreo. Los más relevantes para este trabajo son: el muestreo ideal, el muestreo natural y el muestreo instantáneo (siendo estos últimos los proporcionados por el Sample \& Hold).

\subsubsection{Muestreo Ideal}

A continuación se muestra el desarrollo en el dominio de la frecuencia:
El muestreo ideal es un modelo teórico que supone la multiplicación de la señal $x(t)$ por un tren de
impulsos de Dirac:
\begin{equation}
    x^*(t) = x(t) \cdot \delta_T(t)
\end{equation}
donde $T_s$ es el período de muestreo y $\omega_s = 2\pi/T_s$ la frecuencia de muestreo. 

\begin{equation}
    X^*(\omega) = \frac{1}{2\pi} X(\omega) \otimes \mathcal{F}\{\delta_T(t)\}
\end{equation}

\begin{equation}
    = \frac{1}{T} X(\omega) \otimes \left\{ \omega_s \sum_{m=-\infty}^{\infty} \delta(\omega - m\omega_s) \right\}
\end{equation}

\begin{equation}
    = \frac{1}{T} X(\omega) \otimes \left\{ \sum_{m=-\infty}^{\infty} \delta(\omega - m\omega_s) \right\}
\end{equation}

\begin{equation}
    X^*(\omega) = \frac{1}{T} \sum_{m=-\infty}^{\infty} X(\omega - m\omega_s)
\end{equation}


Donde $\otimes$ denota convolución. 


En el dominio de la frecuencia, este proceso genera réplicas del espectro original $X(\omega)$ desplazadas en múltiplos de $\omega_s$:
Si se cumple el criterio de Nyquist, dichas réplicas no se solapan, y la señal puede recuperarse
exactamente mediante un filtro pasa bajos ideal. 

\bigskip

Este modelo no es realizable en la práctica, ya que no existen impulsos de duración infinitesimal ni amplitud infinita.

\subsubsection{Muestreo Natural}
El muestreo natural consiste en permitir el paso de la señal durante un intervalo finito de tiempo $\tau$, repitiendo este proceso cada $T_s$. Matemáticamente, se describe como:
En el muestreo natural, a diferencia del muestreo ideal, la señal de entrada $x(t)$ no se multiplica 
por un tren de impulsos de Dirac, sino por un tren de pulsos rectangulares de ancho $\tau$. 
De esta manera, la señal muestreada se obtiene como:

\begin{equation}
    x_n(t) = x(t) \cdot p_T(t),
\end{equation}

donde $p_T(t)$ es un tren de pulsos rectangulares de período $T_s$ y ancho $\tau$.

\bigskip

En el dominio de la frecuencia, la multiplicación en el tiempo se transforma en una convolución:

\begin{equation}
    X_n(\omega) = \frac{1}{2\pi} \, X(\omega) \otimes P_T(\omega),
\end{equation}

donde $P_T(\omega)$ es el espectro del tren de pulsos periódicos. 

\bigskip

El espectro del tren de pulsos se puede expresar como una serie de impulsos modulados por una función
$\mathrm{sinc}$:

\begin{equation}
    P_T(\omega) = 2\pi\frac{\tau}{T_s} \sum_{n=-\infty}^{\infty} 
    \mathrm{sinc}\!\left(\frac{n\omega_s \tau}{2}\right) \, \delta(\omega - n\omega_s),
\end{equation}

donde $\omega_s = \tfrac{2\pi}{T_s}$ es la frecuencia de muestreo en radianes. 

\bigskip

Sustituyendo en la convolución:

\begin{align}
    X_n(\omega) 
    &= \frac{1}{2\pi} X(\omega) \otimes 
       \left[ \frac{2\pi}{T_s} \sum_{n=-\infty}^{\infty} 
       \mathrm{sinc}\!\left(\frac{n\omega_s \tau}{2}\right) \, \delta(\omega - n\omega_s) \right] \\[6pt]
    &= \frac{\tau}{T_s} \sum_{n=-\infty}^{\infty} 
       \mathrm{sinc}\!\left(\frac{n\omega_s \tau}{2}\right) \, X(\omega - n\omega_s).
\end{align}

\bigskip

Es decir, el muestreo natural también genera réplicas del espectro $X(\omega)$ en múltiplos de $\omega_s$, 
pero ahora cada réplica está atenuada por un factor de tipo \textit{sinc} que depende tanto de $m$ como del ancho $\tau$ 
del pulso. Pero para un n dado, la sinc permanece constante y no depende de omega

\subsubsection{Muestreo Instantáneo (Zero-Order Hold)}

El bloque \textit{Zero-Order Hold} (ZOH) es el encargado de retener el valor de cada muestra
durante un período de muestreo $T_s$. A diferencia del muestreo natural o ideal, aquí 
cada muestra se transforma en un pulso constante de duración $T_s$. 

\bigskip

La respuesta impulsiva del ZOH es un pulso rectangular de ancho $T_s$:

\begin{equation}
    h_{ZOH}(t) = 
    \begin{cases}
        1, & 0 \leq t < T_s \\
        0, & \text{otro caso}
    \end{cases}
\end{equation}

Su transformada de Fourier es:

\begin{equation}
    H_{ZOH}(\omega) = T_s \, \text{sinc}\!\left(\frac{\omega T_s}{2}\right)
\end{equation}

\bigskip

Si la entrada al bloque es la señal muestreada ideal $x^*(t)$, la salida se obtiene por convolución:

\begin{equation}
    y(t) = x^*(t) \otimes h_{ZOH}(t),
\end{equation}

lo cual en frecuencia se traduce en:

\begin{equation}
    Y(\omega) = X^*(\omega) \cdot H_{ZOH}(\omega).
\end{equation}

Recordando que el espectro del muestreo ideal es:

\begin{equation}
    X^*(\omega) = \frac{1}{T_s} \sum_{n=-\infty}^{\infty} X(\omega - n\omega_s),
\end{equation}

la salida del ZOH resulta:

\begin{equation}
    Y(\omega) = \frac{1}{T_s} \sum_{n=-\infty}^{\infty} 
    X(\omega - n\omega_s) \cdot T_s \, \text{sinc}\!\left(\frac{\omega T_s}{2}\right).
\end{equation}

Ahora la atenuación de la sinc no se mantiene constante para cada réplica, 
sino que depende de $\omega$ para la misma replica.

\subsection{Filtros en el Proceso de Muestreo}
Con el objetivo de eliminar las réplicas de alta frecuencia producidas por el muestreo, es necesario implementar filtros pasabajos.
\subsubsection{Filtro AntiAliasing}
El filtro anti-aliasing se coloca antes del zero-order hold y su función es limitar el espectro de la señal de entrada para que no 
haya componentes de frecuencias mayores que la mitad de la frecuencia de muestreo (cumpliendo así con el criterio de Nyquist \ref{Nyquist}).
Si bien el filtro asegura una reconstrucción más fiel a la señal original, límita las señales de entrada posibles al sistema.
\subsubsection{Filtro de Reconstrucción}
El filtro de reconstrucción se coloca al final del proceso de muestreo y su función es eliminar las réplicas de alta frecuencia generadas por el muestreo.
La freuencia de corte de este filtro pasabajos debe ser mayor a la frecuencia máxima de la señal original, pero menor a la frecuencia de Nyquist.
\subsection{Llave Analógica}
El dispositivo encargado de realizar el muestreo de la señal es el Sample \& Hold, que permite realizar muestreos 
tanto naturales como instantáneos. Para seleccionar el modo de operación, se utiliza una llave analógica (sincronizada
con el duty y la frecuencia del Sample \& Hold) que permite el paso de la señal muestreada en el momento correspondiente al modo elegido.
