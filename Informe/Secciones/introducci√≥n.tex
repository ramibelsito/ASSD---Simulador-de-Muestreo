En esta experiencia de laboratorio se desarrolló un sistema capaz de realizar
muestreo (instantáneo y natural) de señales analógicas. Para ello, se utilizaron varios bloques
que permitirán mayor control y entendimiento sobre el procesamiento de la señal y sus efectos.
De esta forma se logró observar los distintos comportamientos de la señal en cada etapa del
proceso de muestreo.
Además, se implementó el trabajo tanto en una PCB como en simulación en LTSpice y en un entorno
propio desarrllado en Python, lo que permitió comparar los efectos de los distintos componentes
y las diferencias entre los resultados ideales, los modelados y los reales.