En esta práctica se logró implementar y analizar un sistema 
completo de muestreo de señales analógicas, incluyendo las 
modalidades ideal, natural e instantánea (Zero-Order Hold). 
A través de la combinación de simulaciones en Python, LTSpice y 
la implementación en placa, se pudo observar de manera comparativa cómo 
cada tipo de muestreo afecta la señal y su espectro en frecuencia.

Se verificó experimentalmente el Teorema de Nyquist-Shannon, 
confirmando que la frecuencia de muestreo debe ser al menos el 
doble de la frecuencia máxima de la señal para evitar aliasing. 
Además, se evidenció la importancia de los filtros anti-aliasing y 
de reconstrucción, ya que su correcta implementación permite preservar 
la mayor cantidad de armónicos de la señal original, garantizando una 
reconstrucción fiel.


En conclusión, esta experiencia permitió consolidar conceptos teóricos de muestreo, 
procesamiento de señales y filtrado, y desarrollar habilidades prácticas para 
diseñar y analizar sistemas de adquisición de señales analógicas en un entorno 
tanto simulado como real.